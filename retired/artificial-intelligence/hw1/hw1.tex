% LaTeX Article Template - customizing header and footer
\documentclass{article}

\newtheorem{thm}{Theorem}

% Set left margin - The default is 1 inch, so the following
% command sets a 1.25-inch left margin.
\setlength{\oddsidemargin}{0.25in}

% Set width of the text - What is left will be the right margin.
% In this case, right margin is 8.5in - 1.25in - 6in = 1.25in.
\setlength{\textwidth}{6in}

% Set top margin - The default is 1 inch, so the following
% command sets a 0.75-inch top margin.
\setlength{\topmargin}{-0.25in}

% Set height of the header
\setlength{\headheight}{0.3in}

% Set vertical distance between the header and the text
\setlength{\headsep}{0.2in}

% Set height of the text
\setlength{\textheight}{9in}

% Set vertical distance between the text and the
% bottom of footer
\setlength{\footskip}{0.1in}

% Set the beginning of a LaTeX document
\usepackage{multirow}
\usepackage{fullpage}
\usepackage{graphicx}
\usepackage{amsthm}
\usepackage{amssymb}
\usepackage{amssymb}
\usepackage{algpseudocode}
\usepackage{listings}
\usepackage{color}

\definecolor{dkgreen}{rgb}{0,0.6,0}
\definecolor{gray}{rgb}{0.5,0.5,0.5}
\definecolor{mauve}{rgb}{0.58,0,0.82}

\lstset{frame=tb,
  language=python,
  aboveskip=3mm,
  belowskip=3mm,
  showstringspaces=false,
  columns=flexible,
  basicstyle={\small\ttfamily},
  numbers=none,
  numberstyle=\tiny\color{gray},
  keywordstyle=\color{blue},
  commentstyle=\color{dkgreen},
  stringstyle=\color{mauve},
  breaklines=true,
  breakatwhitespace=true,
  tabsize=3
}

\graphicspath{%
    {converted_graphics/}% inserted by PCTeX
    {/}% inserted by PCTeX
}
%%%%%%%%%%%%%%%%%%%%%%%%%%%%%

\begin{document}\title{Homework $1$\\ Computer Science \\ Spring 2017\\ B351}         % Enter your title between curly braces
\author{Steven Myers}        % Enter your name between curly braces
\date{\today}          % Enter your date or \today between curly braces
\maketitle


% Redefine "plain" pagestyle
\makeatother     % `@' is restored as a "non-letter" character


% Set to use the "plain" pagestyle
\pagestyle{plain}
All the work herein is mine.

\section*{Answers}

\begin{enumerate}
    \item % problem one
    \begin{enumerate}
        \item % a
        The state space is as large as there are valid "moves" that the robot can make. We can consider every location within the maze as a "state." We can make some assummptions, such as the maze consists of only rectangular corridors and that the robot moves roughly from square to square within this maze. We could then take every tile in the maze that is not blocked (a wall), and sum the number of possible moves on unblocked tiles that would not result the robot moving into a blocked tile. If necessary, we must also dictate what orientation, and moreover how much distance can be covered from each state as well.
        \item % b
        If we only move until we reach intersections of corridors, it means that we can join long stretches of  tiles in which there are no intersections and treat it as one state, since we are no longer interested in states that do not break a cardinal direction (e.g something that only travels north).
        \item % c
        This does little to change the overall problem of exiting the maze with our robot. The orientation of the robot is identical to the last action taken. So, if the last time we had a transition from one state to another, our action was to move east, then the Robot's orientation would also be east.
        \item % d
        We made the following simplifications:
        \begin{itemize}
            \item
            A robot cannot stop moving until it reaches an intersection.
            \item
            We do not know the size of maze or whether or not the robot can successfully navigate itself out.
            \item
            We do not know the shape of the maze, but assume it consists of rectangular walls.
        \end{itemize}

    \end{enumerate}
    \item % problem two
    A problem in which an AI is not well suited for could be any of the following:
    \begin{itemize}
        \item
        Determining the best tasting wine in a wine-tasting contest
        \item
        Picking out the prettiest dress in a wardrobe
        \item
        Determining whose life is more valuable than another (Will Smith's family in \textit{iRobot})
    \end{itemize}
    \item % problem three
        \begin{enumerate}
            \item
            \lstset{language=python}
            Here is a representation of the floor in a graph \textit{G = \{V, E\}}:
            \begin{lstlisting}[frame=single]
G = [V, E]
V = [U, A, B, H, C, I, J, G, F, D, K, E, L]
E = [(U, C), (U, L), (A, B), (B, H), (B, J), (H, G), (C, I), (J, G), (G, K),
    (F, E), (F, D), (D, K), (D, L)]
            \end{lstlisting}
            \item
            Here is an adjacency list representation of the floor:
            \begin{lstlisting}
adj_list = {
    "U" : ["C", "L"],
    "A" : ["B"],
    "B" : ["A", "H", "J"],
    "H" : ["B", "G"],
    "C" : ["U", "I"],
    "I" : ["C"],
    "J" : ["B", "G"],
    "G" : ["H", "J", "K"],
    "F" : ["E", "D"],
    "D" : ["F", "K", "L"],
    "K" : ["G", "D"],
    "E" : ["F"],
    "L" : ["U", "D"],
}
            \end{lstlisting}
            \item
            Here is one possible sequence of a DFS search on the floor:
            \\\textit{U, C, I, L, D, F, E, K, G, J, B, A, H}
            \item
            Since we know that it takes 4 minutes to scan every room, we can multiply
            the number of minutes times the number of vertices to find the total amount of time
            spent scanning as a constant value. So, it will take 48 minutes to scan the entire
            floor since there are 12 rooms. Since there is only one path to travel from \textbf{U} to \textbf{I}, we can optimize
            the path by starting with our drone in room I. This also guarantees that we don't have to make a roundtrip between \textbf{C} \& \textbf{L}. We will have to consider that our drone is going to have to revisit some rooms to gain access to other parts of the floor. Starting from \textbf{I}, here is the cheapest path and its cost:
            \\\\\textit{\textbf{Path: \{I, C, L, D, F, E, F, D, K, G, J, B, A, B, H\}}}
            \\\textit{\textbf{Cost Breakdown:}} 48 (scanning) + 7 (trip from C to L) + 26 (all other trips) = 81 minutes.
            \\\\ To annotate this graph, I would add edges with numbers between each room indicating the cost of traveling from one room to another. Additionally, I would make it clear in a key off to the side that scanning each room takes 2 minutes.
            \item
            At least two batteries are needed for the drone to scan the entire floor if the drone starts in room \textbf{I}.
            \item
            My python file is "p1.py". My DFS function returned the following path for a DFS search. It is quite similar to the one mentioned in part (c) above, but the python program visited \textbf{H} first instead of \textbf{J}.:
            \begin{lstlisting}
['U', 'C', 'I', 'L', 'D', 'F', 'E', 'K', 'G', 'H', 'B', 'A', 'J']
            \end{lstlisting}
        \end{enumerate}
    \item % problem four
    \begin{enumerate}
        \item I played only rock for my games against the computer.
        The expectation of a human winning is 38\%. I am not surprised at this,
        since there is a one in three (33\%) chance of winning, losing, or tying.
        The computer's expectation is 30\%, which is also roughly 33\%. My results are listed below:
        \begin{lstlisting}
cw% = 60, hm% = 30, ties% = 10
cw% = 10, hm% = 60, ties% = 30
cw% = 40, hm% = 30, ties% = 30
cw% = 20, hm% = 40, ties% = 40
cw% = 20, hm% = 30, ties% = 50
        \end{lstlisting}
        \item
        The python random function selects a number based on the Mersenne Twister pseudo-random number generator.
        The computer's choice of numbers have no effect on the previous choice. For my selection against the computer,
        my previous choice did not necessarily alter my choice; but in real-life against a person, it would have.
        \item
        My version of Robby is in p2.py. I managed to make it such that Robby wins every game by setting the seed of the random generator.

    \end{enumerate}

\end{enumerate}
% \begin{enumerate}
% \item $f(x) = x^2$ \begin{verbatim}$f(x) = x^2$\end{verbatim}
% \item $g(x,y) = \frac{x - 1}{y + 2}$
% \end{enumerate}
%
% To bold, \begin{verbatim} \textbf{robot} \end{verbatim} \textbf{robot}
% To Italicize, \begin{verbatim} \textit{robot} \end{verbatim} \textit{robot}

\end{document}
