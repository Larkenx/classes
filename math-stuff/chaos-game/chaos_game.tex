% LaTeX Article Template - customizing header and footer
\documentclass{article}

\newtheorem{thm}{Theorem}

% Set left margin - The default is 1 inch, so the following
% command sets a 1.25-inch left margin.
\setlength{\oddsidemargin}{0.25in}

% Set width of the text - What is left will be the right margin.
% In this case, right margin is 8.5in - 1.25in - 6in = 1.25in.
\setlength{\textwidth}{6in}

% Set top margin - The default is 1 inch, so the following
% command sets a 0.75-inch top margin.
\setlength{\topmargin}{-0.25in}

% Set height of the header
\setlength{\headheight}{0.1in}

% Set vertical distance between the header and the text
\setlength{\headsep}{0.2in}

% Set height of the text
\setlength{\textheight}{9in}

% Set vertical distance between the text and the
% bottom of footer
\setlength{\footskip}{0.1in}

% Set the beginning of a LaTeX document
\usepackage{multirow}
\usepackage{fullpage}
\usepackage{graphicx}
\usepackage{amsthm}
\usepackage{amssymb}
\usepackage{amssymb}
\usepackage{algpseudocode}
\usepackage{caption}
\usepackage{float}
\usepackage{subcaption}
\graphicspath{%
    {converted_graphics/}% inserted by PCTeX
    {/}% inserted by PCTeX
}
%%%%%%%%%%%%%%%%%%%%%%%%%%%%%




\begin{document}\title{Chaos Game Analysis\\ Spring 2017\\ Math-M330}         % Enter your title between curly braces
\author{Steven Myers}        % Enter your name between curly braces
\date{\today}          % Enter your date or \today between curly braces
\maketitle


% Redefine "plain" pagestyle
\makeatother     % `@' is restored as a "non-letter" character

% Set to use the "plain" pagestyle
\pagestyle{plain}

\section*{Introduction to the Chaos Game}

\paragraph{}
The chaos game is an iterative process that will produce an image known as the Sierpenski triangle. The classification of images, like the Sierpenski triangle generated by the chaos game, is a \textit{fractal}. A fractal is an image that is created by a mathematical process. To demonstrate what is meant by an iterative process, the rules of how to play the chaos game are listed below:
\begin{enumerate}
    \item
    Start by placing three points on a piece of paper. For the best results, try placing the points roughly in the three vertices of an equilateral triangle.
    \item
    Next, select an initial point contained within the area of the vertices, or ontop of the invisibile edges connecting the three verticies you drew.
    \item
    Select one of the three points of the triangle randomly. (You may use a dice or random number generator)
    \item
    Draw a new point halfway between the randomly selected vertice and your initial starting point.
    \item
    Repeat step 3 until a pattern emerges or until satisfied.
\end{enumerate}

\section*{Observing the Chaos Game Pattern}
\paragraph{}
In the first iterations of the chaos game, it's likely that the generated points fall along defined edges leading towards the three vertices. One early pattern that one may notice is that the points will follow a "tug-of-war" pattern, moving back and forth between just two the vertices and forming what appears to be a defined edge. You may even begin to notice an upside down triangle scribed within our original area between the three vertices. Using a computer program, we can generate more points than what would be possible by pencil and paper. In Figure 1, we can view a few resulting images of the chaos game, generated by a computer program.
\begin{figure}[H]
    \includegraphics[width=\linewidth, height=.2\textheight]{combined_image}
    \caption{Images created by the Chaos Game for 50, 250, and 1000 iterations respectively.}
\end{figure}
\paragraph{}
Looking at Figure 1, we can immediately see that there are well-defined spaces where no points fall. When we iterate 250 times or more, we begin to see these spaces fully develop as upside down, circumscribed triangles. So, by iteratively working the chaos game out, we see that a common pattern emerges that is \textit{not} random. We have consistent areas in which no points fall, and if we run the chaos game over and over, we will eventually get the same result regardless of our initial starting point.
\paragraph{}






\end{document}
