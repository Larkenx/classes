% LaTeX Article Template - customizing header and footer
\documentclass{article}

\newtheorem{thm}{Theorem}

% Set left margin - The default is 1 inch, so the following
% command sets a 1.25-inch left margin.
\setlength{\oddsidemargin}{0.25in}

% Set width of the text - What is left will be the right margin.
% In this case, right margin is 8.5in - 1.25in - 6in = 1.25in.
\setlength{\textwidth}{6in}

% Set top margin - The default is 1 inch, so the following
% command sets a 0.75-inch top margin.
\setlength{\topmargin}{-0.25in}

% Set height of the header
\setlength{\headheight}{0.3in}

% Set vertical distance between the header and the text
\setlength{\headsep}{0.2in}

% Set height of the text
\setlength{\textheight}{9in}

% Set vertical distance between the text and the
% bottom of footer
\setlength{\footskip}{0.1in}

% Set the beginning of a LaTeX document
\usepackage{multirow}
\usepackage{fullpage}
\usepackage{graphicx}
\usepackage{amsthm}
\usepackage{amssymb}
\usepackage{amssymb}
\usepackage{algpseudocode}
\graphicspath{%
    {converted_graphics/}% inserted by PCTeX
    {/}% inserted by PCTeX
}
%%%%%%%%%%%%%%%%%%%%%%%%%%%%%




\begin{document}\title{Chaos Game Analysis\\ Math \\ Spring 2017\\ M330}         % Enter your title between curly braces
\author{Steven Myers}        % Enter your name between curly braces
\date{\today}          % Enter your date or \today between curly braces
\maketitle


% Redefine "plain" pagestyle
\makeatother     % `@' is restored as a "non-letter" character

% Set to use the "plain" pagestyle
\pagestyle{plain}

\section*{Introduction to the Chaos Game}

The chaos game is an iterative process that will produce an image known as the Sierpenski triangle. The classification of images, such as the one generated by the chaos game, is a \textit{fractal}. A fractal is an image that is created by a mathematical process. Before we talk more about fractals, let us first clearly denote the rules of the chaos game. \\\\To play the chaos game:
\begin{enumerate}
    \item
    Start by placing three points on a piece of paper. For the best results, try placing the points roughly in the three vertices of an equilateral triangle.
    \item
    Next, select an initial point contained within the area of the vertices, or ontop of the invisibile edges connecting the three verticies you drew.
    \item
    Select one of the three vertices randomly. (You may use a dice or random number generator)
    \item
    Draw a new point halfway between the randomly selected vertice and your initial starting point.
    \item
    Repeat step 3.
\end{enumerate}

\end{document}
