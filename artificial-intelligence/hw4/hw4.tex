% LaTeX Article Template - customizing header and footer
\documentclass{article}

\newtheorem{thm}{Theorem}

% Set left margin - The default is 1 inch, so the following
% command sets a 1.25-inch left margin.
% \setlength{\oddsidemargin}{0.05in}

% Set width of the text - What is left will be the right margin.
% In this case, right margin is 8.5in - 1.25in - 6in = 1.25in.
\setlength{\textwidth}{8in}

% Set top margin - The default is 1 inch, so the following
% command sets a 0.75-inch top margin.
\setlength{\topmargin}{-0.25in}

% Set height of the header
\setlength{\headheight}{0.3in}

% Set vertical distance between the header and the text
\setlength{\headsep}{0.2in}

% Set height of the text
\setlength{\textheight}{8.5in}

% Set vertical distance between the text and the
% bottom of footer
\setlength{\footskip}{0.1in}

% Set the beginning of a LaTeX document
\usepackage{multirow}
\usepackage{fullpage}
\usepackage{graphicx}
\usepackage{amsthm}
\usepackage{amssymb}
\usepackage{amssymb}
\usepackage{algpseudocode}
\usepackage{listings}
\usepackage{color}
\usepackage{float}
\usepackage{enumitem}

\definecolor{dkgreen}{rgb}{0,0.6,0}
\definecolor{gray}{rgb}{0.5,0.5,0.5}
\definecolor{mauve}{rgb}{0.58,0,0.82}

\lstset{frame=tb,
  language=python,
  aboveskip=3mm,
  belowskip=3mm,
  showstringspaces=false,
  columns=flexible,
  basicstyle={\small\ttfamily},
  numbers=none,
  numberstyle=\tiny\color{gray},
  keywordstyle=\color{blue},
  commentstyle=\color{dkgreen},
  stringstyle=\color{mauve},
  breaklines=true,
  breakatwhitespace=true,
  tabsize=3
}

\graphicspath{%
    {converted_graphics/}% inserted by PCTeX
    {/}% inserted by PCTeX
}
%%%%%%%%%%%%%%%%%%%%%%%%%%%%%

\begin{document}\title{Homework $4$\\ Computer Science \\ Spring 2017\\ B351}         % Enter your title between curly braces
\author{Steven Myers}        % Enter your name between curly braces
\date{\today}          % Enter your date or \today between curly braces
\maketitle

% Redefine "plain" pagestyle
\makeatother     % `@' is restored as a "non-letter" character

% Set to use the "plain" pagestyle
\pagestyle{plain}
All the work herein is mine.

\section*{Answers}

\begin{enumerate}
    \item % problem one
    \begin{enumerate}
        \item % a
        \begin{enumerate}
            \item $\exists\ y\ p(y) \vee [\exists\ y\ (q(y) \rightarrow (\exists\ x\ (p(x) \vee\ q(x,y,C)))]$
            \item $\exists\ y\ p(y) \vee [\exists\ y\ (\neg q(y) \vee (\exists\ x\ (p(x) \vee\ q(x,y,C)))]$ $\rightarrow$ replacement
            \item $\exists\ y\ p(y) \vee [\exists\ z\ (\neg q(z) \vee (\exists\ x\ (p(x) \vee\ q(x,z,C)))]$ standardizing variables
            \item $p(A) \vee \neg q(F(z)) \vee\ p(G(x)) \vee\ q(G(x),F(z),C)$ skolemize $\exists$ to constants and functions
            \item $[[[p, A], [\neg, q, F(z)], [p, G(x)], [q, G(x), F(z), C]]]$ Pythonic Representation\\
        \end{enumerate}

        \item % b
        \begin{enumerate}
            \item $\forall x \forall y \forall x\ d(x,y) \wedge d(y,z) \rightarrow d(x,z)$
            \item $\forall x \forall y \forall x\ d(x,y) \wedge \neg d(y,z) \vee d(x,z)$ $\rightarrow$ elimination
            \item $d(x,y) \wedge \neg d(y,z) \vee d(x,z)$ drop universal quantifiers
            \item $(d(x,y) \vee d(x,z)) \wedge (\neg d(y,z) \vee d(x,z))$ distribute $\vee$ over $\wedge$
            \item $[[[d,x,y],[d,x,z]], [[\neg,d,y,z],[d,x,z]]]$ Pythonic Representation\\
        \end{enumerate}

        \item % c
        \begin{enumerate}
            \item $( P \vee Q) \wedge (\neg P \rightarrow (Q \vee R))$
            \item $( P \vee Q) \wedge (P \vee Q \vee R)$ $\rightarrow$ elimination and double negation
            \item $[[[P], [Q]], [[P], [Q], [R]]]$ Pythonic Representation\\
        \end{enumerate}


    \end{enumerate}

    \item % problem two
    \begin{enumerate}
        \item \textit{True}. $m(1,1) \wedge m(2,1) \wedge m(3, 2)$ is true, so the universal quantifier is true in all possible values of 'x'.
        \item \textit{False}. There is no ordered pair in the set of $m$ ordered pairs such that the number 3 is the second element in the ordered pair, thus not all instances of the $\forall x$ are true.
        \item \textit{False}. In the languages I am aquainted with, $x$ is not a free variable, so this is syntactically incorrect. Moreover, if it were correct syntactically, we would apply the rules by the main connective, so we would have something like $\exists x\ m(1,1) \wedge \exists x\ m(2, 2) \wedge \exists x\ m(3, 3)$ which is also false.
        \item \textit{False}. There is no value for $x$ such that it is the first component of every ordered pair in the set of ordered pairs $m$.
        \item \textit{False}. There is no value for $x$ such that it is the second component of every ordered pair in the set of ordered pairs $m$.
        \item \textit{True}. Again, see note for \textit{c}. Here, $\exists x$ is our main connective, so we get $\forall x\ m(1,1) \vee \forall x\ m(2, 2) \vee \forall x\ m(3, 3)$ and this is true since $m(1,1)$ is true.
    \end{enumerate}
    \item
    \begin{enumerate}
        \item \textit{True}. One in three times $p(x)$ evaluates to \textit{false}, and two in three times it is \textit{true}. So, it is exactly twice as often.
        \item \textit{True}. Expands out to $My m(y,1) \rightarrow p(1) \wedge My m(y,2) \rightarrow p(2) \wedge My m(y,3) \rightarrow p(3)$
    \end{enumerate}
    % \pagebreak
    \item % four
    The domain is not specified, so I will assume the domain is the set of dogs Ursala, Kaiser, and Shilah. In plain English, the FOL sentence states that there is some dog, call that dog $x$, and another dog, call that dog $y$. $x$ is gray and $y$ is silver, and $x$ loves $y$. In order to do resolution refutation, we must convert the sentence to CNF or clausal form. Additionally, we'll need to translate our three premises into the formal language. Since we have the three dogs, we'll need an individual constant for each one.\\\\
    \textit{\textbf{Interpretation (Constants and Domain)}}\\
    \begin{tabular}{l}
        $Domain = \{Ursala, Kaiser, Shilah\}$\\
        $U = Ursala$\\
        $K = Kaiser$\\
        $S = Shilah$\\
    \end{tabular}\\\\
    \textit{\textbf{Premises (converted to FOL then to CNF)}}\\
    \begin{tabular}{lll}
        a. & $silver(U)$\\
        b. & $gray(S) \wedge loves(S, K)$\\
        c. & 1. $\neg(gray(K) \leftrightarrow silver(K)) \wedge loves(K, U)$ \\
           & 2. $(gray(K) \wedge \neg silver(K)) \vee (silver(K) \wedge \neg gray(K)) \wedge loves(K, U)$ & $\leftrightarrow$ equivalence rule\\

    \end{tabular}\\\\
    \textit{\textbf{Conclusion (converted to CNF)}}\\
    \begin{tabular}{ll}
        $\exists x \exists y (gray(x) \wedge silver(y) \wedge loves(x,y))$\\
        $\exists y (gray(F(x)) \wedge silver(y) \wedge loves(F(x),y))$ & Skolemize or bind $\exists x$ to skolem function $F(x)$\\
        $gray(F(x)) \wedge silver(G(y)) \wedge loves(F(x),G(y))$ & Skolemize or bind $\exists y$ to skolem function $G(y)$\\
    \end{tabular}\\\\
    \textit{\textbf{Resolution}}\\
    \begin{tabular}{lll}
        1. &  $silver(U)$ & Premise\\
        2. & $gray(S) \wedge loves(S, K)$ & Premise\\
        3. & $(gray(K) \wedge \neg silver(K)) \vee (silver(K) \wedge \neg gray(K)) \wedge loves(K, U)$ & Premise\\
        4. & $\neg (gray(F(x)) \wedge silver(G(y)) \wedge loves(F(x),G(y)))$ & Conclusion\\
        5. & $\neg gray(F(x)) \vee \neg silver(G(y)) \vee \neg loves(F(x), G(y))$ & from 4\\
        6. & $silver(U) \wedge (\neg gray(F(x)) \vee \neg silver(G(y)) \vee \neg loves(F(x), G(y)))$& Addition on lines 1, 5\\
        7. & $silver(U) \wedge (\neg gray(F(x)) \vee \neg silver(U) \vee \neg loves(F(x), U))$ & Let $G(y) = U$ and substitute \\
        8. & $(silver(U) \wedge \neg gray(Fx)) \vee (silver(U) \wedge loves(F(x), U)\ \vee$& Distribute $\wedge$ over $\vee$\\
           & $ (silver(U) \wedge \neg silver(U))$\\\\

        9. & $(silver(U) \wedge \neg gray(Fx)) \vee (silver(U) \wedge loves(F(x), U))$ & Resolve the last disjunct\\
        10. & ... &\\
    \end{tabular}
    \pagebreak
    \item % 5
    \textit{\textbf{Interpretation (Constants and Domain)}}\\
    \begin{tabular}{l}
        $Domain =$ All Objects\\
        $pushable(x) =$ x is pushable\\
        $red(x) =$ x is red\\
        $green(x) =$ x is green\\
        $C = Object 1$ (A cart)\\
        $O = Object 2$ (A pile of ore)\\
    \end{tabular}\\\\
    \textit{\textbf{Translations}}
    \begin{enumerate}
        \item If pushable objects are green, the non-pushable are red.\\
        \begin{tabular}{lll}
            i. & $\forall x (pushable(x) \rightarrow green(x)) \rightarrow \forall y (\neg pushable(y) \rightarrow red(y))$ & FOL\\
            ii. & $\exists x(pushable(x) \wedge \neg green(x)) \vee \forall y(pushable(y) \vee red(y))$ & $\rightarrow$ elimination and $\neg\neg$ elimination\\
            iii. & $(pushable(G(x)) \wedge \neg green(G(x)) \vee pushable(y) \vee red(y)$& Drop $\forall y$ and skolemize $\exists x$ with $G(x)$\\
            iiii. & $[[[[pushable, G(x)], [\neg, green, G(x)]], [pushable,y], [red,y]]]$ & Pythonic Representation\\
        \end{tabular}
        \item All objects are either green or red.\\
        \begin{tabular}{lll}
            i. & $\forall x (green(x) \vee red(x))$ & FOL\\
            ii. & $green(x) \vee red(x)$ & CNF\\
            iii. & $[[[green,x],[red,x]]]$  & Pythonic Representation\\
        \end{tabular}\\\\
        \textit{\textbf{*This problem does not state exclusivity of red and green objects}}
        \item Object 1, a cart, is pushable.\\
        \begin{tabular}{lll}
            i. & $pushable(C)$ & FOL and CNF\\
            ii. & $[[pushable, C]]$  & Pythonic Representation
        \end{tabular}\\

        \item Object 2, a pile of ore, is not pushable.\\
        \begin{tabular}{lll}
            i. & $\neg pushable(O)$ & FOL and CNF\\
            ii. & $[[pushable, O]]$  & Pythonic Representation
        \end{tabular}\\\\
        \textit{\textbf{Resolution Refutation}}\\
        \begin{tabular}{lll}
            i. & $\neg \exists x(red(x))$ & $\neg$ Conclusion\\
            ii. & $\neg red(x)$ & Drop universal quantifier\\
            iii. & $[[pushable, O],[pushable, C],[[green,x],[red,x]],$ & Premises and Conclusion\\
                & $[[[pushable, G(x)], [\neg, green, G(x)]], [pushable,y], [red,y]], [\neg, red, x]]$
        \end{tabular}\\\\
    \end{enumerate}

    \item % 6
    \begin{itemize}
        \item
        \begin{enumerate}
            \item Cathy is a mechanic.
            \item $M(C)$
        \end{enumerate}
        \item
        \begin{enumerate}
            \item Bob is not a mechanic.
            \item $\neg M(B)$
        \end{enumerate}
        \item
        \begin{enumerate}
            \item Either Alex is a mechanic or Bob is, but I know Cathy works at NASA.
            \item $(M(A) \vee M(B)) \wedge N(C)$
        \end{enumerate}
        \item
        \begin{enumerate}
            \item Bob introduced Alex to Cathy, since Cathy works at NASA.
            \item $I(B,A,C) \wedge N(C)$ \textit{See https://legacy.earlham.edu/~peters/courses/log/transtip.htm}
        \end{enumerate}
        \item
        \begin{enumerate}
            \item Someone is a mechanic, but everyone works at NASA.
            \item $\exists x M(x) \wedge \forall y N(y)$
        \end{enumerate}
        \item
        \begin{enumerate}
            \item Bob introduced himself to Cathy.
            \item $I(B, B, C)$
        \end{enumerate}
        \item
        \begin{enumerate}
            \item Nobody as been introduced to Alex.
            \item $\forall x \forall y \neg I(X, Y, A)$
        \end{enumerate}
        \item
        \begin{enumerate}
            \item If someone introduced Bob to Alex, then Bob isn't a mechanic.
            \item $\exists x I(x, B, A) \rightarrow \neg M(B)$
        \end{enumerate}
        \item
        \begin{enumerate}
            \item Nobody works with anyone here.
            \item $\forall x \forall y \neg W(x, y)$
        \end{enumerate}
        \item
        \begin{enumerate}
            \item Somebody works with Cathy, but it's not a mechanic, because Cathy works at NASA.
            \item $\exists x (W(x, C) \wedge \neg M(x)) \wedge N(C)$
        \end{enumerate}



    \end{itemize}



\end{enumerate}
\end{document}
