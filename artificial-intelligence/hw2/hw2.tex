% LaTeX Article Template - customizing header and footer
\documentclass{article}

\newtheorem{thm}{Theorem}

% Set left margin - The default is 1 inch, so the following
% command sets a 1.25-inch left margin.
\setlength{\oddsidemargin}{0.25in}

% Set width of the text - What is left will be the right margin.
% In this case, right margin is 8.5in - 1.25in - 6in = 1.25in.
\setlength{\textwidth}{6in}

% Set top margin - The default is 1 inch, so the following
% command sets a 0.75-inch top margin.
\setlength{\topmargin}{-0.25in}

% Set height of the header
\setlength{\headheight}{0.3in}

% Set vertical distance between the header and the text
\setlength{\headsep}{0.2in}

% Set height of the text
\setlength{\textheight}{9in}

% Set vertical distance between the text and the
% bottom of footer
\setlength{\footskip}{0.1in}

% Set the beginning of a LaTeX document
\usepackage{multirow}
\usepackage{fullpage}
\usepackage{graphicx}
\usepackage{amsthm}
\usepackage{amssymb}
\usepackage{amssymb}
\usepackage{algpseudocode}
\usepackage{listings}
\usepackage{color}
\usepackage{float}


\definecolor{dkgreen}{rgb}{0,0.6,0}
\definecolor{gray}{rgb}{0.5,0.5,0.5}
\definecolor{mauve}{rgb}{0.58,0,0.82}

\lstset{frame=tb,
  language=python,
  aboveskip=3mm,
  belowskip=3mm,
  showstringspaces=false,
  columns=flexible,
  basicstyle={\small\ttfamily},
  numbers=none,
  numberstyle=\tiny\color{gray},
  keywordstyle=\color{blue},
  commentstyle=\color{dkgreen},
  stringstyle=\color{mauve},
  breaklines=true,
  breakatwhitespace=true,
  tabsize=3
}

\graphicspath{%
    {converted_graphics/}% inserted by PCTeX
    {/}% inserted by PCTeX
}
%%%%%%%%%%%%%%%%%%%%%%%%%%%%%

\begin{document}\title{Homework $1$\\ Computer Science \\ Spring 2017\\ B351}         % Enter your title between curly braces
\author{Steven Myers}        % Enter your name between curly braces
\date{\today}          % Enter your date or \today between curly braces
\maketitle


% Redefine "plain" pagestyle
\makeatother     % `@' is restored as a "non-letter" character


% Set to use the "plain" pagestyle
\pagestyle{plain}
All the work herein is mine.

\section*{Answers}

\begin{enumerate}
    \item % problem one
    Here are the definitions for the terms listed in Problem 3.10:
    \begin{enumerate}
        \item \textbf{State} - A particular assignment of value to all variables for a given situation.
        \item \textbf{State space} - All of the possible assignments of values to variable in a scenario.
        \item \textbf{Search tree} - An arrangement of states such that no state is repeated twice when traversing all states.
        \item \textbf{Search node} - A particular node in a search tree that has a cost and ingoing or outgoing paths. Each node contains the cost of the current node, its data (state values), and the cost of visiting other accessible states.
        \item \textbf{Goal} - A final search node with an ideal or target state that we want to obtain to resolve a problem.
        \item \textbf{Action} - An operation that results in a change of state.
        \item \textbf{Transition model} - The process of moving from one state to another when an action is applied.
        \item \textbf{Branching factor} - The maximum number of "branches" or pathways that some search node can have. This is used to limit the search space and prevent a machine from running out of memory before resolving a problem when the search space is too large.
    \end{enumerate}
    \item % Describe a state space in which iterative deepening search performs much worse than depth-first search (for example, O(n2) vs. O(n)).
    A state space where iterative deepning search would perform worse than DFS is one in which the following are true:
    \begin{enumerate}
        \item The branching factor is large.
        \item The depth of each branch is shallow, and the goal node exists in a branch which is deeper than the others.
    \end{enumerate}
    A state space such as Figure 1 could produce a circumstance where DFS would perform significantly better than iterative deepening search.
    \begin{figure}[H]
        \centering
        \includegraphics[width=.5\linewidth, height=.15\textheight]{dfs_example}
        \caption{Attach a goal node \textbf{S} as a node \textbf{O}'s right child. (Image taken from textbook figure 3.16 on page 86, Russel \& Norvig)}
    \end{figure}
    \item Yes, the graph is consistent. Here are the statements for each node and its sucessors:
    \begin{enumerate}
        \item AB: $1 \leq 6$
        \item AC: $1 \leq 8$
        \item BC: $4 \leq 4$
    \end{enumerate}
    \item
\end{enumerate}
\end{document}
